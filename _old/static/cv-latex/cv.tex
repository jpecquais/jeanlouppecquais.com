\documentclass[11pt,a4paper,sans]{moderncv}

\moderncvstyle{casual}
\moderncvcolor{blue}

\setlength{\hintscolumnwidth}{2cm}
\usepackage[utf8]{inputenc}% encodage, à modifier selon vos habitudes
\usepackage[scale=0.8]{geometry}% pour régler les marges du CV les options habituelles de l'extension geometry peuvent s'appliquer ici
\usepackage{helvet}% pour utiliser la police helvetica par exemple.
\usepackage[french]{babel}% pour un document en français.

\name{Jean-Loup}{Pecquais}% no comment
\title{Ingénieur du son / Formateur}% ceci est optionnel et permet d'ajouter des informations en dessous du titre du CV. À commenter si on veut l'enlever.
\address{21 rue de Paris}{91\,120 Palaiseau}{France}% optionnel aussi, supprimer ou laisser vide l'argument pays par exemple.
%les données suivantes sont aussi optionnelles donc à commenter si on n'en veut pas
\phone[mobile]{06~76~15~78~86}
\email{jeanlouppecquais@ik.me}
\homepage{www.jeanlouppecquais.com}
%\social[linkedin]{pierre.durand}
%\social[twitter]{pierre.durand}
%\social[github]{pierre.durand}
%\extrainfo{informations complémentaires.}
%\quote{Encore un titre}%toujours optionnel, se place avant le corps du CV

\begin{document}
\makecvtitle
\section{Formation}
\cventry{Depuis\,2022}{Conservatoire National des Arts et Métiers}{diplôme visé : Ingénieur systèmes électroniques}{}{\textit{Paris}}{}
\cventry{2016--2019}{\'Ecole nationale supérieure Louis Lumière}{Master Son}{}{\textit{Saint-Denis}}{}
\cventry{2014--2016}{\'Ecole préparatoire}{Institut Saint Stanislas}{}{\textit{Nîmes}}{}
\cventry{2011--2012}{Baccalauréat Série S}{Lycée Saint Joseph}{}{\textit{Carpentras}}%
    {\textit{Spécialité physique}}% on peut mettre ici de 3 à 6 arguments qui peuvent être laissés vides

\section{Experience professionnelle}

\cventry{Depuis\,2022}{Intervenant à l'ENS Louis Lumière}{Saint-Denis}{}{}{%
\begin{itemize}
\item Infomatique audio ;
\item Utilisation du logiciel REAPER ;
\item Initiation à la programmation (lua \& FAUST).
\end{itemize}}

\cventry{Depuis\,2021}{Formateur à Whiti Audio Formation}{Saint-Jean le Blanc}{}{}{%
\begin{itemize}
\item Initiation 3D sonore ;
\item Utilisation du logiciel REAPER ;
\item Prise de son \& mixage.
\end{itemize}}

\cventry{Depuis\,2020}{Consultant pour FLUX:: Immersive}{Meung-sur-Loire}%
            {}%
            {}
{%
\begin{itemize}%
\item  \'Elaboration et enseignement de la formation sur le logiciel SPAT Revolution;
\item Documentation de l'ensemble des produits;
\item Développement de l'extension ReaVolution pour REAPER;
\end{itemize}}

\cventry{Depuis\,2020}{Formateur à Transversal Studio}{Vannes}{}{}{%
\begin{itemize}
\item Prise de son \& mixage.
\item Utilisation du logiciel REAPER ;
\end{itemize}}

\section{Langues}
%Possibilité d'insérer des commentaires dans les entrées
\cvitemwithcomment{Anglais}{Lu, parlé, écrit}{}
\section{Compétences informatiques}
%possibilité de mettre les entrées en deux colonnes 
\cvdoubleitem{REAPER}{Avancé}{SPAT\,Rev.}{Avancé}
\cvdoubleitem{C++}{Intermédiaire}{Python}{Intermédiaire}
\cvdoubleitem{FAUST}{Intermédiaire}{Max/Pd}{Intermédiaire}
\section{Centres d'intérêts}
\cvitem{Hobby}{Guitariste et chanteur des groupe Outwards et Pompéi}
%\cvitem{Hobby 2}{Description}
\end{document}